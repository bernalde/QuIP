\documentclass[11pt, a4paper]{article}
%\usepackage{geometry}
\usepackage[inner=1.5cm,outer=1.5cm,top=2.5cm,bottom=2.5cm]{geometry}
\pagestyle{empty}
\usepackage{graphicx}
\usepackage{fancyhdr, lastpage, bbding, pmboxdraw}
\usepackage[usenames,dvipsnames]{color}
\definecolor{darkblue}{rgb}{0,0,.6}
\definecolor{darkred}{rgb}{.7,0,0}
\definecolor{darkgreen}{rgb}{0,.6,0}
\definecolor{red}{rgb}{.98,0,0}
\usepackage[colorlinks,pagebackref,pdfusetitle,urlcolor=darkblue,citecolor=darkblue,linkcolor=darkred,bookmarksnumbered,plainpages=false]{hyperref}
\renewcommand{\thefootnote}{\fnsymbol{footnote}}

\pagestyle{fancyplain}
\fancyhf{}
\lhead{ \fancyplain{}{47-779. Quantum Integer Programming (QuIP)} }
%\chead{ \fancyplain{}{} }
\rhead{ \fancyplain{}{CMU Fall 2020} }
%\rfoot{\fancyplain{}{page \thepage\ of \pageref{LastPage}}}
\fancyfoot[RO, LE] {page \thepage\ of \pageref{LastPage} }
\thispagestyle{plain}

%%%%%%%%%%%% LISTING %%%
\usepackage{listings}
\usepackage{caption}
\DeclareCaptionFont{white}{\color{white}}
\DeclareCaptionFormat{listing}{\colorbox{gray}{\parbox{\textwidth}{#1#2#3}}}
\captionsetup[lstlisting]{format=listing,labelfont=white,textfont=white}
\usepackage{verbatim} % used to display code
\usepackage{fancyvrb}
\usepackage{acronym}
\usepackage{amsthm}
\VerbatimFootnotes % Required, otherwise verbatim does not work in footnotes!

\usepackage[square,sort,numbers]{natbib}
\bibliographystyle{unsrtnat}



\definecolor{OliveGreen}{cmyk}{0.64,0,0.95,0.40}
\definecolor{CadetBlue}{cmyk}{0.62,0.57,0.23,0}
\definecolor{lightlightgray}{gray}{0.93}



\lstset{
%language=bash,                          % Code langugage
basicstyle=\ttfamily,                   % Code font, Examples: \footnotesize, \ttfamily
keywordstyle=\color{OliveGreen},        % Keywords font ('*' = uppercase)
commentstyle=\color{gray},              % Comments font
numbers=left,                           % Line nums position
numberstyle=\tiny,                      % Line-numbers fonts
stepnumber=1,                           % Step between two line-numbers
numbersep=5pt,                          % How far are line-numbers from code
backgroundcolor=\color{lightlightgray}, % Choose background color
frame=none,                             % A frame around the code
tabsize=2,                              % Default tab size
captionpos=t,                           % Caption-position = bottom
breaklines=true,                        % Automatic line breaking?
breakatwhitespace=false,                % Automatic breaks only at whitespace?
showspaces=false,                       % Dont make spaces visible
showtabs=false,                         % Dont make tabls visible
columns=flexible,                       % Column format
morekeywords={__global__, __device__},  % CUDA specific keywords
}

%%%%%%%%%%%%%%%%%%%%%%%%%%%%%%%%%%%%
\begin{document}
\begin{center}
{\Large \textsc{47-779. Quantum Integer Programming}}
\end{center}
\begin{center}
Mini-1, Fall 2020
\end{center}
%\date{September 26, 2014}


\begin{center}
\rule{6in}{0.4pt}
\begin{minipage}[t]{.75\textwidth}
\begin{tabular}{llr}
\textbf{Room:} Zoom Online & & \textbf{Time:} Tuesday and Thursday 5:20pm-7:10pm
\end{tabular}
\end{minipage}
\rule{6in}{0.4pt}
\end{center}
\setlength{\unitlength}{1in}
\renewcommand{\arraystretch}{2}


\noindent\textbf{Instructors:}
\begin{center}
\rule{6in}{0.4pt}
\begin{minipage}[t]{.75\textwidth}
\begin{tabular}{lll}
Sridhar Tayur & \textbf{Email:}  \href{mailto:stayur@cmu.edu}{stayur@cmu.edu} & \textbf{Office:} 4216 Tepper Quad \\
\hline
Davide Venturelli & \textbf{Email:}  \href{mailto:DVenturelli@usra.edu}{DVenturelli@usra.edu} & \textbf{Office:} Online \\
\hline
David E. Bernal & \textbf{Email:}  \href{mailto:bernalde@cmu.edu}{bernalde@cmu.edu} & \textbf{Office:} 3116 Doherty Hall
\end{tabular}
\end{minipage}
\rule{6in}{0.4pt}
\end{center}


%\noindent\textbf{Course Pages:} \begin{enumerate}
%\item \url{https://bernalde.github.io/QuIP/}
%\end{enumerate}

\vskip.15in
\noindent\textbf{Office Hours:} Post your questions in the forum provided for this purpose on Canvas. This course will be conducted online.


\vskip.15in
\noindent\textbf{Objectives:}  This course is primarily designed for graduate students (and advanced undergraduates) interested in integer programming (with non-linear objective functions) and the potential of near-term quantum and quantum-inspired computing for solving combinatorial optimization problems.
By the end of the semester, someone enrolled in this course should be able to:
\begin{itemize}
    \item Appreciate the current status of quantum computing and its potential use for integer programming
    \item Access and use quantum computing resources (such as D-Wave Quantum Annealers)
    \item Set up a given integer program to be solved with quantum computing 
    \item Work in groups collaboratively on a state-of-the-art project involving applications of quantum computing and integer programming
\end{itemize}

\noindent This course is not going to focus on the following topics:
\begin{itemize}
    \item Quantum Gates and Circuits
    \item Computational complexity theory
    \item Quantum Information Theory
    \item Analysis of speedup using differential geometry, algebraic topology, etc.
\end{itemize}

\vskip.15in
\noindent\textbf{Prerequisite classes and capabilities:}
Although this class has no explicit prerequisites we consider a list of recommended topics and skills that the student should feel comfortable with.
An undergraduate-level understanding of probability, calculus, statistics, graph theory, algorithms, and linear algebra is assumed.
Knowledge of linear and integer programming will be useful for this course.
Programming skills are strongly recommended.
Basic concepts in physics are recommended but lack of prior knowledge is not an issue as pertinent ones will be covered in the lectures.
No particular knowledge in quantum mechanics or algebraic geometry is required. 

\vspace*{.15in}

Students with backgrounds in operations research, industrial engineering, chemical engineering, electrical engineering, physics, computer science, or applied mathematics are strongly encouraged to consider taking this course.
\newpage

%\vspace*{.25in}

\noindent \textbf{Tentative Course Outline:}
\begin{center} 
\begin{minipage}{5in}
\begin{flushleft}
{\bf Part 1 - Integer programming (classical methods)} \dotfill ~1 week \\
{\color{darkred}{\Rectangle}} ~Integer Programming basics \cite{conforti2014integer}. \\
{\color{darkred}{\Rectangle}} ~Cutting plane theory and relaxations \cite{conforti2014integer}. \\
{\color{darkred}{\Rectangle}} ~Introduction to Test Sets  \cite{sturmfels1996grobner,tayur1995algebraic}. \\
~~~{\color{darkred}{\Rectangle}} ~ Gr{\"o}bner basis \cite{bertsimas2000new,hocsten1995grin}. \\
~~~{\color{darkred}{\Rectangle}} ~ Graver basis \cite{hemmecke2011polynomial}. \\

{\bf Part 2 - Ising, QUBO} \dotfill ~1 week \\
%(Unconstrained methods, penalty methods)\\
{\color{darkred}{\Rectangle}} ~Ising model basics \cite{brush1967history,sherrington1975solvable,ray1989sherrington}. \\
{\color{darkred}{\Rectangle}} ~Simulated Annealing \cite{kirkpatrick1983optimization,koulamas1994survey}. \\
{\color{darkred}{\Rectangle}} ~Markov-chain Monte Carlo methods \cite{metropolis1953equation,bortz1975new,troyer2005computational,young2008size}. \\
{\color{darkred}{\Rectangle}} ~Benchmarking classical methods \cite{dunning2018works,coffrin2019evaluating}. \\
{\color{darkred}{\Rectangle}} ~Formulating combinatorial problems as QUBOs \cite{lucas2014ising}. \\


{\bf Part 3 - GAMA: Graver Augmented Multiseed algorithm} \dotfill ~1 week \\
{\color{darkred}{\Rectangle}} ~GAMA \cite{alghassi2019graver}. \\
%Part 6 - Applications \dotfill ~1 week \\
~~~{\color{darkred}{\Rectangle}} ~Applications: Portfolio Optimization \cite{alghassi2019graver}, Cancer Genomics \cite{alghassi2019quantum} \\
~~~{\color{darkred}{\Rectangle}} ~Quantum Inspired: Quadratic (Semi-)Assignment Problem \cite{alghassi2019gama}.


{\bf Part 4 - Hardware for solving Ising/QUBO} \dotfill ~1 week \\
{\color{darkred}{\Rectangle}} ~Graphical Processing Units \cite{yavorsky2019highly,cook2019gpu,romero2019performance}. \\
{\color{darkred}{\Rectangle}} ~Tensor Processing Units \cite{yang2019high}. \\
{\color{darkred}{\Rectangle}} ~Complementary metal-oxide-semiconductors (CMOS) \cite{yamaoka201520k}. \\
{\color{darkred}{\Rectangle}} ~Digital Annealers \cite{aramon2019physics}. \\
{\color{darkred}{\Rectangle}} ~Oscillator Based Computing \cite{chou2019analog,wang2019oim}. \\
{\color{darkred}{\Rectangle}} ~Coherent Ising Machines \cite{roques2020heuristic,inagaki2016coherent,king2018emulating,hamerly2019experimental,tiunov2019annealing,mcmahon2016fully}. \\

{\bf Part 5 - Quantum methods for solving Ising/QUBO} \dotfill ~1 week \\
{\color{darkred}{\Rectangle}} ~AQC, Quantum Annealing and D-Wave \cite{mcgeoch2020theory,albash2018adiabatic,das2008colloquium,santoro2006optimization,farhi2001quantum,kadowaki1998quantum,johnson_quantum_2011}. \\
{\color{darkred}{\Rectangle}} ~QAOA: Quantum Approximate Optimization Algorithm \cite{farhi2014quantum,hadfield2017quantum,hadfield2019quantum}. \\


{\bf Part 6 - Other topics and project presentations} \dotfill ~1 week \\
{\color{darkred}{\Rectangle}} ~Compiling \\
~~~{\color{darkred}{\Rectangle}} ~Quantum Annealing \cite{bernal2019integer,dridi2018novel}. \\
~~~{\color{darkred}{\Rectangle}} ~Gate-based  Noisy Intermediate Scale Quantum (NISQ) devices \cite{dridi2019knuth}. \\
{\color{darkred}{\Rectangle}} ~Adiabatic Quantum Computing and Algebraic Geometry \\
~~~{\color{darkred}{\Rectangle}} Minimizing Polynomial Functions \cite{dridi2019minimizing}.\\
~~~{\color{darkred}{\Rectangle}} Prime Factorization \cite{dridi2017prime}.
\end{flushleft}
\end{minipage}
\end{center}

\vspace*{.15in}
\noindent\textbf{Grading Policy:} Weekly homework and quizzes (50\%), Final Group Project (50\%).
\begin{itemize}
    \item Each lecture will have a short quiz to evaluate the concepts covered in a previous class. The two worst quizzes won't be counted towards the final grade.
    \item The group final project will require the implementation and solution of an integer programming instance. This final project deliverable is a report and a presentation.
\end{itemize}

\vspace*{.15in}
\noindent\textbf{Project description:} The final project accounts for 50\% of the total grade. This project will be completed in groups of 2-4 students and will reflect the understanding of the students of the material covered in the lecture. The components of this project are the following
\begin{itemize}
    \item Identify a problem that can be posed as an Integer Program. Discuss the importance of this problem.
    \item Solve instances of the identifies problem using classical tools. Identify which are the sources of complexity while solving this problem.
    \item Model the problem as a Quadratic Unconstrained Binary Optimization (QUBO). Verify that the reformulation of the problem is valid, in the sense that it represents the original problem.
    \item Solve the resulting QUBO using non-conventional methods, e.g. Quantum Annealing, QAOA, simulated annealing in GPUs/TPUs, etc. Compare at least two different methods.
    \item Write a report outlining the different approaches used and highlighting the knowledge obtained while developing the project.
    \item Hold a final presentation in front of the class reporting the findings of the project.
\end{itemize}


\vskip.15in
\noindent\textbf{Important Dates:}
\begin{center} \begin{minipage}{3.8in}
\begin{flushleft}
No Final Exam. Presentations in weeks 6-7.  \\
\end{flushleft}
\end{minipage}
\end{center}

\vskip.15in
\noindent\textbf{Course Policy:}  
\begin{itemize}
\item Auditing students are encouraged to participate actively in the lectures.
\item Regular attendance is essential and expected.
\item Please sign up using Canvas.
\end{itemize}

\vskip.15in
\noindent\textbf{Highlights:}
\begin{enumerate}
    \item Development of this course was supported by the Air Force Research Lab NYSTEC-USRA Contract (FA8750-19-3-6101)
    \item Access to D-Wave systems might be available via written proposals to the University Space Research Association (USRA). See \url{https://riacs.usra.edu/quantum/rfp} for terms and conditions. The course will discuss proposal preparation.
    \item Students of this course are encouraged to apply to the Feynman Academy Internship program \url{https://riacs.usra.edu/quantum/qacademy} that sponsors research projects at NASA Ames Research Center.
\end{enumerate}

\vskip.15in
\noindent\textbf{Academic Honesty:}   Lack of knowledge of the academic honesty policy is not a reasonable explanation for a violation.
Any form of plagiarism can earn you a failing grade for the entire course.
For more information you can refer to \href{https://www.cmu.edu/policies/student-and-student-life/academic-integrity.html}{CMU’s policies on academic integrity}. When in doubt, add a citation.


\vskip.15in
\noindent\textbf{Casual References:} %\footnotemark
There is no single text book for the course. This is a short list of various interesting and useful books that will be mentioned during the course. You need to consult them occasionally.
\begin{itemize}
\item Georges Irfah, {\textit {The Universal History of Computing}}, John Wiley \& Sons, 2001.
\item A. Das and B.K. Chakrabarti (Eds.). {\textit {Quantum Annealing and Related Optimization Methods}, Springer-Verlag, 2005}.
\item Eleanor G. Rieffel and Wolfgang H. Polak, {\textit{Quantum Computing: A Gentle Introduction}}, MIT Press, 2011.
\item Richard J. Lipton and Kenneth W. Regan, {\textit{Quantum Algorithms via Linear Algebra. A Primer}}, MIT Press, 2014.
\end{itemize} 

\vskip.15in
\noindent\textbf{Student Resources:}
If you have a disability and require accommodations, please contact Catherine Getchell, Director of Disability Resources, 412-268-6121, \href{mailto:getchell@cmu.edu}{getchell@cmu.edu}.
If you have an accommodations letter from the Disability Resources office, we encourage you to discuss your accommodations and needs with us as early in the semester as possible.
We will work with you to ensure that accommodations are provided as appropriate.

As a student, you may experience a range of challenges that can interfere with learning, such as strained relationships, increased anxiety, substance use, feeling down, difficulty concentrating and/or lack of motivation.
These mental health concerns or stressful events may diminish your academic performance and/or reduce your ability to participate in daily activities.
CMU services are available, and treatment does work.
You can learn more about confidential mental health services available on campus at: \url{http://www.cmu.edu/counseling/}.
Support is always available (24/7) from Counseling and Psychological Services: 412-268-2922.




% \footnotetext{Downloadable ebook versions are available on Canvas.}
%%%%%% THE END 

% BibTeX users please use one of
%\bibliographystyle{spbasic}      % basic style, author-year citations
%\bibliographystyle{spmpsci}      % mathematics and physical sciences
%\bibliographystyle{spphys}       % APS-like style for physics
%\bibliography{}   % name your BibTeX data base
\bibliography{reference}
\end{document} 